\documentclass[11pt,brazil, a4paper]{article}
\usepackage[top=2cm, bottom=2cm, left=2cm, right=2cm]{geometry}
\usepackage{color}
\usepackage{geometry}
\usepackage[portuges]{babel}
\usepackage[T1]{fontenc}
\usepackage[ansinew]{inputenc}
\usepackage{array}
\usepackage{colortbl}
\usepackage{epsfig}
\usepackage{multirow}
\usepackage{verbatim}
\usepackage{listings}

\begin{document}

\selectlanguage{brazil}
\lstset{language=C++}

\begin{tabular*}{18cm}{p{2cm}p{12cm}@{\extracolsep{\fill}}}
 \multirow{2}{2cm}{\epsfig{file=../../lib/puclogo,width=1.5cm}}
  &  \vspace{1mm} \sf PONTIFICIA UNIVERSIDADE CAT�LICA DE MINAS GERAIS \\
  & \sf Departamento de Ci�ncia da Computa��o \\
  &  \sf Cora��o Eucar�stico \vspace{3mm}  \\
\end{tabular*}
\begin{tabular*}{17.2cm}{|p{6cm}|p{5.5cm}|p{2cm}|p{1.95cm}|}
\hline
\sf Disciplina & \sf Curso & \sf Turno & \sf Per�odo \\

\multicolumn{1}{|l|}{Algor�tmos e Estruturas de Dados I} &
\multicolumn{1}{c|}{Ci�ncia da Computa��o} &
\multicolumn{1}{c|}{Manh�} &
\multicolumn{1}{l|}{1$^\circ$} \\ \hline \hline
\multicolumn{4}{|l|}{\sf Professor} \\
\multicolumn{4}{|l|}{Felipe Cunha (felipe@pucminas.br)} \\ \hline
\end{tabular*}

\vspace{0.5cm}

\begin{center}
{\Large \bf Lista de Exerc�cios 05}
\end{center}

\begin{enumerate}
	\item Fazer um m�todo recursivo que recebe um n�mero inteiro e retorna o seu fatorial.
	
	
	\item Fazer um m�todo recursivo que recebe um n�mero inteiro n e retorna o n-�simo termo da equa��o de recorr�ncia abaixo:

	$T(1) = 2$

	$T(2) = 3$

	$T(n) = 5 * n + T(n - 1)^{n}$


	\item Fazer um m�todo recursivo que recebe um n�mero inteiro e positivo n e calcula o somat�rio abaixo.

	$n + (n-1) + ... + 1 + 0$

	\item Fazer um m�todo recursivo que imprima de um n�mero natural em base bin�ria
	
	\item Fazer um m�todo recursivo que multiplique dois n�meros naturais, atrav�s de somas sucessivas
	
	\item Fazer um m�todo recursivo que calcule o MDC (m�ximo divisor comum) de dois inteiros positivos $m$ e $n$
	
	\item Fazer um m�todo recursivo que conte os d�gitos de um determinado n�mero.
	
	\item Fazer um m�todo recursivo que determine se um n�mero � ou n�o primo.
	
	
\end{enumerate}

\end{document}