\documentclass[11pt,brazil, a4paper]{article}
\usepackage[top=2cm, bottom=2cm, left=2cm, right=2cm]{geometry}
\usepackage{color}
\usepackage{geometry}
\usepackage[portuges]{babel}
\usepackage[T1]{fontenc}
\usepackage[ansinew]{inputenc}
\usepackage{array}
\usepackage{colortbl}
\usepackage{epsfig}
\usepackage{multirow}
\usepackage{verbatim}
\usepackage{listings}
\usepackage{multicol}

\begin{document}

\selectlanguage{brazil}
\lstset{language=C++}

\begin{tabular*}{18cm}{p{2cm}p{12cm}@{\extracolsep{\fill}}}
 \multirow{2}{2cm}{\epsfig{file=../../lib/puclogo,width=1.5cm}}
  &  \vspace{1mm} \sf PONTIFICIA UNIVERSIDADE CAT�LICA DE MINAS GERAIS \\
  & \sf Departamento de Ci�ncia da Computa��o \\
  &  \sf Cora��o Eucar�stico \vspace{3mm}  \\
\end{tabular*}
\begin{tabular*}{17.2cm}{|p{6cm}|p{5.5cm}|p{2cm}|p{1.95cm}|}
\hline
\sf Disciplina & \sf Curso & \sf Turno & \sf Per�odo \\

\multicolumn{1}{|l|}{Algor�tmos e Estruturas de Dados I} &
\multicolumn{1}{c|}{Ci�ncia da Computa��o} &
\multicolumn{1}{c|}{Manh�} &
\multicolumn{1}{l|}{1$^\circ$} \\ \hline \hline
\multicolumn{4}{|l|}{\sf Professor} \\
\multicolumn{4}{|l|}{Felipe Cunha (felipe@pucminas.br)} \\ \hline
\end{tabular*}

\vspace{0.5cm}

\begin{center}
{\Large \bf Lista de Exerc�cios 07 - Files}
\end{center}

\begin{enumerate}
	\item Suponha que temos dois arquivos cujo as linhas s�o ordenadas lexigraficamente. Por exemplo, estes arquivos podem conter nomes de pessoas, linha a linha, em ordem alfab�tica. Escreva um programa que receba, por linha de comando, os nomes destes dois arquivos e de um terceiro. Crie este terceiro arquivo contendo todas as linhas
destes dois arquivos ordenadas lexicograficamente.
	\begin{multicols}{2}
	\begin{verbatim}
	Arquivo 1:
	Antonio
	Berenice
	Diana
	Solange
	Sonia
	Zuleica

	Arquivo 2:
	Carlos
	Celia
	Fabio
	Henrique

	Arquivo resultante:
	Antonio
	Berenice
	Carlos
	Celia
	Diana
	Fabio
	Henrique
	Solange
	Sonia
	Zuleica
	\end{verbatim}
	\end{multicols}
	
	\item Escreva um programa em C para contar a quantidade de palavras de um arquivo texto.
	
	\item Escreva um programa em C que abra um arquivo texto e que conte a quantidade de caracteres armazenados nele. Imprima o n�mero na tela. O programa deve solicitar ao usu�rio que digite o nome do arquivo.
	
	\item Escreva um programa em C que solicite ao usu�rio a digita��o do nome de um arquivo texto j� existente, e que ent�o gere um outro arquivo, que ser� uma c�pia do primeiro.
	
	\item Considere um arquivo texto que armazene n�meros em ponto flutuante em cada uma de suas linhas. Escreva um programa em C que determine o valor m�ximo, o valor m�nimo e a m�dia desses valores armazenados no arquivo. Imprima esses valores na tela.

\end{enumerate}

\end{document}